%This is my super simple Real Analysis Homework template

\documentclass[leqno]{article}
\usepackage[utf8]{inputenc}
\usepackage[english]{babel}
\usepackage[]{amsthm} %lets us use \begin{proof}
\usepackage{amsmath}
\usepackage[]{amssymb} %gives us the character \varnothing

\title{Homework 3}
\author{Aline Bessa and Rao Li}
\date\today
%This information doesn't actually show up on your document unless you use the maketitle command below

\begin{document}
\maketitle %This command prints the title based on information entered above

%Section and subsection automatically number unless you put the asterisk next to them.
\section*{Question 1} TBD

\hfill

\section*{Question 2} Waiting for Rao's answer

\hfill


\section*{Question 3} Waiting for Rao's answer 

\hfill

\section*{Question 4} \textbf{(a)} When there are two classes, the maximum entropy occurs when they are equaly likely, i.e., 
when \textit{half} of the examples are positive and \textit{half} are negative. The closer the proportions are to $\frac{1}{2}$, 
the higher the entropy. In the first dataset, we have that the proportions for positive and negative class are, respectively, 
$\frac{4}{9}$ and $\frac{5}{9}$. Consequently, the difference between these proportions and $\frac{1}{2}$ are the same, and can 
be calculated as
\begin{equation*}
\begin{split}
&|\frac{4}{9} - \frac{1}{2}| = \frac{|2*4 - 9*1|}{18} = \frac{1}{18}
\end{split}
\end{equation*}
As for the second dataset, the proportions for positive and negative class are, respectively, $\frac{1}{3}$ and $\frac{2}{3}$. 
The difference between these proportions and $\frac{1}{2}$ are the same, and can be calculates as 
\begin{equation*}
\begin{split}
&|\frac{1}{3} - \frac{1}{2}| = \frac{|2*1 - 3*1|}{6} = \frac{1}{6}
\end{split}
\end{equation*}
Given that the difference for the first dataset is smaller, its entropy is higher. In other words, the entropy for the dataset with 4 positive and 
5 negative examples is higher.

\hfill

\noindent \textbf{(b)} 
\end{document}

