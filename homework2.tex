%This is my super simple Real Analysis Homework template

\documentclass[leqno]{article}
\usepackage[utf8]{inputenc}
\usepackage[english]{babel}
\usepackage[]{amsthm} %lets us use \begin{proof}
\usepackage{amsmath}
\usepackage[]{amssymb} %gives us the character \varnothing

\title{Homework 2}
\author{Aline Bessa \textbf{and Li Rao}}
\date\today
%This information doesn't actually show up on your document unless you use the maketitle command below

\begin{document}
\maketitle %This command prints the title based on information entered above

%Section and subsection automatically number unless you put the asterisk next to them.
\section*{Question 1}

\textbf{(a)} Let $v$, $a$, and $t$ denote variables for cancer volume, age, and cancer type respectively. We then have that:
\begin{equation*}
\hat{y} = w_0 + w_1 \times v
\end{equation*}
where $w_0$ and $w_1$ are the parameters of model 1 and
\begin{equation*}
\hat{y} = w'_0 + w'_1 \times v + w'_2 \times a
\end{equation*}
where $w'_0$, $w'_1$, and $w'_2$ are the parameters of model 2.

\hfill

\textbf{(b)} Instead of using $t$, which is a categorical variable, we use one-hot encoding and generate variable $x_3$, where 1 stands for cancer type I and 
0 stands for cancer type II. We then have that:
\begin{equation*}
\hat{y} = w''_0 + x_3 \times w''_1 \times v + (1 - x_3) \times w''_2 \times v  + w''_3 \times a 
\end{equation*}
where $w''_0$, $w''_1$, $w''_2$, and $w''_3$ are the parameters of model 3.

\hfill

\textbf{(c)} Model 1 has 2 parameters and model 2 has 3 parameters. Model 2 is more complex because it has more parameters.

\hfill

\textbf{(d)} For model 1, the matrix $X'$ that corresponds to $X$'s first three rows is
\[
X'=
  \begin{bmatrix}
    1 & 0.7 \\
    1 & 1.3 \\
    1 & 1.6
  \end{bmatrix}
\]
where the values correspond to cancer volumes.

For model 2, the matrix $X'$ that corresponds to $X$'s first three rows is
\[
X'=
  \begin{bmatrix}
    1 & 0.7 & 55 \\
    1 & 1.3 & 65 \\
    1 & 1.6 & 70
  \end{bmatrix}
\]
where the first column corresponds to cancer volumes and the second column corresponds to ages.

For model 3, and considering the one-hot encoding suggested in the question, the matrix $X'$ that corresponds to $X$'s first three rows is
\[
X'=
  \begin{bmatrix}
    1 & 0.7 & 0 & 55\\
    1 & 0 & 1.3 & 65 \\
    1 & 0 & 1.6 & 70
  \end{bmatrix}
\]
where the first column corresponds to cancer volumes, the second column corresponds to ages, and the third column corresponds to the one-hot encoding 
of the cancer types.

\hfill

\textbf{(e)} Model 3 should be selected because both its training and test MSE are the lowest.


\hfill

\section*{Question 2} Let $r$, $f$, $t$, $s$, and $y$ be the amount of rainfall, the amount of fertilizer, the temperature, the number of sunny days, and 
crop yield respectively. Then we have that
\begin{equation*}
\hat{y} = w_0 + w_1^{k_1} \times r + w_2^{k_2} \times f + w_3^{k_3} \times t + w_4^{k_4} \times s
\end{equation*}
where $k_1$, $k_2$, $k_3$, and $k_4$ are greater than zero. If all of them are equal to 1, the regression is linear; otherwise, it's polynomial.
\textbf{check with Linda}

\section*{Question 3} \textbf{(a)} To get the final function $g(x) = w_1x + w_0$, we need to calculate coefficients $w_0$ and $w_1$. Using the closed 
formula
\begin{equation*}
\textbf{w} = (X^TX)^-1X^T\textbf{y}
\end{equation*}
where 
\[
X=
  \begin{bmatrix}
    1 & 1050\\
    1 & 428\\
    1 & 362\\
    1 & 529\\
    1 & 790\\
    1 & 401\\
    1 & 380\\
    1 & 1454\\
    1 & 1127\\
    1 & 700
  \end{bmatrix}
\]
and
\[
\textbf{y}=
  \begin{bmatrix}
     57\\
     28\\
     26\\
     40\\
     60\\
     22\\
     38\\
     110\\
     100\\
     46
  \end{bmatrix}
\]
we have that $w_0 = -1.97639231$ and $w_1 = 0.07571859$~\footnote{We used numpy to multiply the matrices and compute the inverse.}.
 Consequently, we have that $g(x) = 0.07571859x + -1.97639231$.

\hfill

\textbf{(b)} By using this function, we get that $g(475) = 33.98993693$, a.k.a., the predicted number of stories is 34.

\end{document}
