%This is my super simple Real Analysis Homework template

\documentclass[leqno]{article}
\usepackage[utf8]{inputenc}
\usepackage[english]{babel}
\usepackage[]{amsthm} %lets us use \begin{proof}
\usepackage{amsmath}
\usepackage[]{amssymb} %gives us the character \varnothing

\title{Homework 1}
\author{Aline Bessa -- N19183671}
\date\today
%This information doesn't actually show up on your document unless you use the maketitle command below

\begin{document}
\maketitle %This command prints the title based on information entered above

%Section and subsection automatically number unless you put the asterisk next to them.
\section*{Question 1}

\textbf{(a)} Let $L$ be a random variable that denotes the location where the rocks were collected, taking on values $\{A, B\}$. We then have that:

\begin{equation*}
P(L = B) = 4 \times P(L = A)
\end{equation*}
and
\begin{equation*}
P(L = A) + P(L = B) = 1 
\end{equation*}

\noindent Consequently, 

\begin{equation*}
\begin{split}
&P(L = A) + 4 \times P(L = A) = 1\\
&5 \times P(L = A) = 1\\
&P(L = A) = 0.2
\end{split}
\end{equation*}
and
\begin{equation*}
\begin{split}
&P(L = B) = 4 \times P(L = A) \\
&P(L = B) = 4 \times 0.2 \\
&P(L = B) = 0.8
\end{split}
\end{equation*}

\noindent \textbf{(b)} Let $w$ be a random variable that represents the weight of rocks. We then have to compute the \textit{posterior probability} that the rocks are from 
location A, i.e., $P(L = A| w_1 = 9.3, w_2 = 8.8, w_3 = 9.8)$, where $w_1, w_2$ and $w_3$ simply indicate 3 samples from $w$. Applying Bayes' theorem, we have that:

\begin{equation}
P(L = A| w_1 = 9.3, w_2 = 8.8, w_3 = 9.8) = \frac{P(w_1 = 9.3, w_2 = 8.8, w_3 = 9.8| L = A) \times P(L = A)}{P(w_1 = 9.3, w_2 = 8.8, w_3 = 9.8)}
\label{eq1}
\end{equation}

\noindent Assuming that the weights of the rocks ($w_1$, $w_2$, $w_3$) are conditionally independent, i.e., that the rock weights in a fixed location are independent,
we can rewrite the numerator in the right side of Equation \ref{eq1} as

\begin{equation*}
\begin{split}
&P(w_1 = 9.3, w_2 = 8.8, w_3 = 9.8| L = A) \times P(L = A) = 
\\
&P(w_1 = 9.3 | L = A) \times P(w_2 = 8.8 | L = A) \times P(w_3 = 9.8| L = A) \times P(L = A)
\end{split}
\end{equation*}

\noindent Let us tackle this numerator first. Previously, we calculated that $P(L = A) = 0.2$. To compute the other factors, we use the probability density~\footnote{I used a scientific calculator to automate the calculation of pdfs throughout this document.} of the 
Gaussian distribution of rock weights in Location A, i.e., we use $\mu_1 = 9.2$ and $\sigma_1 = 1.6$.

\begin{equation*}
\begin{split}
P(w_1 = 9.3 | L = A) = P(w_1 = 9.3 | \mu_1 = 9.2, \sigma_1 = 1.6) = \frac{1}{\sqrt{2\pi(1.6)^2}}e^{-\frac{(9.3 - 9.2)^2}{2(1.6)^2}} \approx 0.246 
\\
P(w_1 = 8.8 | L = A) = P(w_1 = 8.8 | \mu_1 = 9.2, \sigma_1 = 1.6) = \frac{1}{\sqrt{2\pi(1.6)^2}}e^{-\frac{(8.8 - 9.2)^2}{2(1.6)^2}} \approx 0.203
\\
P(w_1 = 9.8 | L = A) = P(w_1 = 9.8 | \mu_1 = 9.2, \sigma_1 = 1.6) = \frac{1}{\sqrt{2\pi(1.6)^2}}e^{-\frac{(9.8 - 9.2)^2}{2(1.6)^2}} \approx 0.232
\end{split}
\end{equation*}
  
\noindent Combining these values with the prior, we have the following value for the numerator in Equation~\ref{eq1}:

\begin{equation*}
\begin{split}
&P(w_1 = 9.3, w_2 = 8.8, w_3 = 9.8| L = A) \times P(L = A) \approx 0.246 \times 0.203 \times 0.232 \times 0.2
\\
&P(w_1 = 9.3, w_2 = 8.8, w_3 = 9.8| L = A) \times P(L = A) \approx 0.000232
\end{split} 
\end{equation*}

\noindent As for the denominator in Equation~\ref{eq1}, it can be rewritten as the following marginal probability:

\begin{equation*}
\begin{split}
&P(w_1 = 9.3, w_2 = 8.8, w_3 = 9.8) = 
\\
&P(w_1 = 9.3, w_2 = 8.8, w_3 = 9.8| L = A) \times P(L = A) + 
\\
&P(w_1 = 9.3, w_2 = 8.8, w_3 = 9.8| L = B) \times P(L = B) 
\end{split}
\end{equation*}

\noindent We already computed the parts of the denominator that concern $L = A$, so we just have to compute the terms that involve $L = B$. We
again also assume conditional independency and use the probability density of the 
Gaussian distribution of rock weights in Location B, i.e., we use $\mu_2 = 9.6$ and $\sigma_2 = 1.2$.

\begin{equation*}
\begin{split}
P(w_1 = 9.3 | L = B) = P(w_1 = 9.3 | \mu_2 = 9.6, \sigma_2 = 1.2) = \frac{1}{\sqrt{2\pi(1.2)^2}}e^{-\frac{(9.3 - 9.6)^2}{2(1.2)^2}} \approx 0.322 
\\
P(w_1 = 8.8 | L = B) = P(w_1 = 8.8 | \mu_2 = 9.6, \sigma_2 = 1.2) = \frac{1}{\sqrt{2\pi(1.2)^2}}e^{-\frac{(8.8 - 9.6)^2}{2(1.2)^2}} \approx 0.266
\\
P(w_1 = 9.8 | L = B) = P(w_1 = 9.8 | \mu_2 = 9.6, \sigma_2 = 1.2) = \frac{1}{\sqrt{2\pi(1.2)^2}}e^{-\frac{(9.8 - 9.6)^2}{2(1.2)^2}} \approx 0.328
\end{split}
\end{equation*}

\noindent Combining these terms with what was already computed for $L = A$, and with prior $P(L = B) = 0.8$, we can rewrite the denominator of Equation~\ref{eq1} as:

\begin{equation*}
\begin{split}
&P(w_1 = 9.3, w_2 = 8.8, w_3 = 9.8) = P(w_1 = 9.3, w_2 = 8.8, w_3 = 9.8| L = A) \times P(L = A) + 
\\
&P(w_1 = 9.3, w_2 = 8.8, w_3 = 9.8| L = B) \times P(L = B)
\\
\\
&P(w_1 = 9.3, w_2 = 8.8, w_3 = 9.8) \approx 0.000232 + P(w_1 = 9.3, w_2 = 8.8, w_3 = 9.8| L = B) \times P(L = B)
\\
%% &P(w_1 = 9.3, w_2 = 8.8, w_3 = 9.8) \approx 0.000232 + (P(w_1 = 9.3 | L = B) \times P(w_2 = 8.8 | L = B) \times P(w_3 = 9.8| L = B) \times P(L = B))
%% \\
&P(w_1 = 9.3, w_2 = 8.8, w_3 = 9.8) \approx 0.000232 +  (0.322 \times 0.266 \times 0.328 \times 0.8)
\\
&P(w_1 = 9.3, w_2 = 8.8, w_3 = 9.8) \approx 0.000232 + 0.0225 
\\
&P(w_1 = 9.3, w_2 = 8.8, w_3 = 9.8) \approx 0.0227
\end{split}
\end{equation*}

\noindent Combining the values computed for numerator and denominator, we have that
\begin{equation*}
\begin{split}
&P(L = A| w_1 = 9.3, w_2 = 8.8, w_3 = 9.8) \approx \frac{0.000232}{0.0227}
\\
&P(L = A| w_1 = 9.3, w_2 = 8.8, w_3 = 9.8) \approx 0.010
\end{split}
\end{equation*}

\noindent This posterior probability indicates that, given the prior and the evidence, it is not very likely that the rocks were collected in Location A.
\textbf{RE-CHECK PDF VALUES!}

\noindent \textbf{(c)} The ML hypothesis is the value $x$ of variable $L$, that maximizes $P(w_1 = 9.3, w_2 = 8.8, w_3 = 9.8| L = x)$. Using the results in
\textbf{(b)}, we have that:

\begin{equation*}
P(w_1 = 9.3, w_2 = 8.8, w_3 = 9.8| L = A) \approx 0.246 \times 0.203 \times 0.232 \approx 0.012
\end{equation*}
and

\begin{equation*}
P(w_1 = 9.3, w_2 = 8.8, w_3 = 9.8| L = B) \approx 0.322 \times 0.266 \times 0.328 \approx 0.028
\end{equation*}

\noindent Given the weights of the rocks, the Gaussian distributions for $L = A$ and $L = B$, and ignoring the priors, we have that
the ML hypothesis is $L = B$, i.e., Location $x = B$ maximizes $P(w_1 = 9.3, w_2 = 8.8, w_3 = 9.8| L = x)$. \textbf{CHECK PDF VALUES}
  
\end{document}
