\documentclass[leqno]{article}
\usepackage[utf8]{inputenc}
\usepackage[english]{babel}
\usepackage[]{amsthm} %lets us use \begin{proof}
\usepackage{amsmath}
\usepackage[]{amssymb} %gives us the character \varnothing
\usepackage{hyperref}
\usepackage{booktabs} % For formal tables
\usepackage[ruled,linesnumbered]{algorithm2e}
\usepackage{url}
%\usepackage{natbib}
\usepackage{xcolor}
\usepackage{subfig}
\let\proof\relax
\let\endproof\relax
\usepackage{enumitem}
\usepackage{xspace}
\usepackage{graphicx}
\usepackage{subfig}
\usepackage{tablefootnote}
\usepackage{balance}
\usepackage{bibunits}

\title{Answers for Programming Part}
\author{Aline Bessa and Li Rao}
\date\today
%This information doesn't actually show up on your document unless you use the maketitle command below

\begin{document}
\maketitle %This command prints the title based on information entered above

%Section and subsection automatically number unless you put the asterisk next to them.
\section*{Programming question 1} \textbf{Waiting on Rao}


\hfill

\section*{Programming question 2}

\subsection*{1a} The answers for items in question 1a are:

\noindent (i) The predicted label is 1.

\hfill

\noindent (ii) The confusion matrix on the test set for $k = 1$ is:

\begin{tabular}{l|l|c|c|}
\multicolumn{2}{c}{}&\multicolumn{1}{c}{predicted \textit{(y)}}\\
\cline{3-4}
\multicolumn{2}{c|}{}&+&-\\
\cline{2-4}
\multirow{correct \textit{(r)}}& + & 209 & 64\\
\cline{2-4}
& - & 134 & 93 \\
\cline{2-4}
\end{tabular}

\hfill\hfill

\noindent (iii) For $k = 1$, we have Accuracy = $\frac{TP + TN}{TP + TN + FP + FN} = \frac{209 + 93}{500} = 0.604$, true positive rate = $\frac{TP}{TP + FN} = \frac{209}{209 + 64} = 0.76556776556$, and false positive rate = $\frac{FP}{FP + TN} = \frac{134}{134 + 93} = 0.59030837004$.

\hfill

\noindent (iv) The predicted label is 1.

\hfill

\noindent (v) The confusion matrix on the test set for $k = 5$ is:

\begin{tabular}{l|l|c|c|}
\multicolumn{2}{c}{}&\multicolumn{1}{c}{predicted \textit{(y)}}\\
\cline{3-4}
\multicolumn{2}{c|}{}&+&-\\
\cline{2-4}
\multirow{correct \textit{(r)}}& + & 212 & 61\\
\cline{2-4}
& - & 136 & 91 \\
\cline{2-4}
\end{tabular}

\hfill\hfill

\noindent (vi) For $k = 5$, we have Accuracy = $\frac{TP + TN}{TP + TN + FP + FN} = \frac{212 + 91}{500} = 0.606$, true positive rate = $\frac{TP}{TP + FN} = \frac{212}{212 + 61} = 0.77655677655$, and false positive rate = $\frac{FP}{FP + TN} = \frac{136}{136 + 91} = 0.59911894273$.

\hfill

\noindent (vii) For $k = 5$, the accuracy is 0.606, as computed above. \textbf{Am I missing something or is this question redundant given the above?}

\hfill

\noindent (viii) The confusion matrix on the test set with Zero-R is:

\begin{tabular}{l|l|c|c|}
\multicolumn{2}{c}{}&\multicolumn{1}{c}{predicted \textit{(y)}}\\
\cline{3-4}
\multicolumn{2}{c|}{}&+&-\\
\cline{2-4}
\multirow{correct \textit{(r)}}& + & 273 & 0\\
\cline{2-4}
& - & 227 & 0 \\
\cline{2-4}
\end{tabular} 


\hfill \hfill

\subsection*{1b} \textbf{do at home}

\subsection*{1c} 

\end{document}
