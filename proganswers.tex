\documentclass[leqno]{article}
\usepackage[utf8]{inputenc}
\usepackage[english]{babel}
\usepackage[]{amsthm} %lets us use \begin{proof}
\usepackage{amsmath}
\usepackage[]{amssymb} %gives us the character \varnothing

\title{Answers for Programming Part}
\author{Aline Bessa -- N19183671}
\date\today
%This information doesn't actually show up on your document unless you use the maketitle command below

\begin{document}
\maketitle %This command prints the title based on information entered above

%Section and subsection automatically number unless you put the asterisk next to them.
\section*{Question 1} The estimated value of $P(C)$ for $C = 1$ is 0.4018006002.

%\hfill

\section*{Question 2} The estimated value of $P(C)$ for $C = 0$ is 0.5981993998.

%\hfill

\section*{Question 3} The estimated values for $\mu$ and $\sigma^2$ for the Gaussian corresponding to attribute $capital\_run\_length\_longest$ and 
class 1 (Spam) are: $\mu$  = 97.209129 and $\sigma^2$ = 36369.991113.

%\hfill

\section*{Question 4}  The estimated values for $\mu$ and $\sigma^2$ for the Gaussian corresponding to attribute $char\_freq\_;$ and 
class 0 are: $\mu$ = 0.048426 and $\sigma^2$  = 0.088306.

%\hfill

\section*{Question 5} For the first 5 examples in the test set, the predicted class was \textit{Non-Spam} (0).

%\hfill

\section*{Question 6} For the last 5 examples in the test set, the predicted class was \textit{Non-Spam} (0).


%\hfill

\section*{Question 7} The percentage error on the test examples is 20\%.

%\hfill

\section*{Question 8} The accuracy attained with Zero-R is of 59\%, whereas the accuracy attained with the Gaussian Naive Bayes is 80\%.

\section*{Question 9}

\end{document}
